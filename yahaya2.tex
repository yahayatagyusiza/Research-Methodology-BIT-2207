\documentclass[12pt,a4paper]{article}
\begin{document}
\section{Introduction}
\subsection{Introduction}
A bilateral trade is the exchange of goods between two countries that facilitates trade and investment by reducing or eliminating tariffs, import quotas, export restraints and other trade barriers.\\
East Africa is a geographically and economically homogeneous region committed to regional integration. This region has a community called the East African Community (EAC) that consists of Burundi, Rwanda, Tanzania, Uganda and Kenya. These countries are divided into two the least Developed Countries which include Burundi, Rwanda, Tanzania and Uganda and the non-Least Developed Countries which include only Kenya. All the countries in the East African Community are members of the World Trade Organization (WTO)
\subsection{Background}
The EAC established a Customs Union in 2005 which was fully-fledged with zero internal tariffs as from 2010. The EAC, in fast tracking its economic integration process, ratified a more far-reaching common market protocol in July 2010. In November 2013, EAC Members signed a protocol on a monetary union.\\
The integration agenda of the EAC is strongly political in nature as its ultimate goal is to become a federation.
The World Trade Organization (WTO) is the only global international organization dealing with the rules of trade between nations. At its heart are the WTO agreements, negotiated and signed by the bulk of the world's trading nations and ratified in their parliaments.\\
Exports to the EU from East African Community are dominated by coffee, cut flowers, tea, tobacco, fish and vegetables.
Imports from the EU into the region are dominated by machinery and mechanical appliances, equipment and parts, vehicles and pharmaceutical products.
\subsection{Problem Statement}
Uganda has been facing problems with trading problems with its neighbors where good are smuggled into and out of Uganda in order to dodge taxes. There haven’t been favorable policies put in place by authorities like the trade ministry and the East African Community in facilitating bilateral trade effectively.\\
Due to lack of favorable policies, bilateral trade in Uganda has faced problems of taxes, smuggling of goods, price fluctuations and poor understanding of the market.
\end{document} 